\subsection{Users}\label{users}
My project is intended for a general audience, not simply for women’s advocates within technology. It will consequently require a strong narrative thread, since viewers may not be particularly interested in the topic and may not know how to interpred analytical visualizations. To aid in designing for a primarily non-specialist audience, I will be designing the narrative, visualization, and interactions with four personas in mind:
\begin{itemize}
  \item Gabriella Griffin, 19, undergraduate psychology major
  \item Ryan Mitchell, 28, software engineer
  \item Laurie Woods, 34, front-end developer/women’s coding meetup organizer
  \item Monica Pierce, 47, accountant/mother of two (Ian, 18; Julie, 16)
\end{itemize}

Gabriella, profiled in the sidebar on the next page, is an excellent example of the project’s target audience: a young woman comfortable using computers, with some programming experience but no exposure to career opportunities in IT\@.
%% TODO: Add Gabriella's profile %%


\subsection{Tasks}\label{tasks}
%% TODO: Convert from plan to description %%
The narrative portion of the project will guide users through some of the key insights in the data; a full exploratory analysis of the data is required to determine what questions the narrative will address. Since the visualization will have four main components (overview, high school, higher education, career), the narrative should introduce each of these areas. This could be accomplished with any of \citet{SegelHeer2010Narrative} narrative structures, depending on the relationship between the final dataset and the users’ needs.

The exploratory portion of the project is shaped by the users’ needs much more directly. It should present the data clearly enough to let them explore their own questions, such as:

\begin{itemize}
  \item What kinds of technical careers are open to women? What kinds are still relatively closed?
  \item When do women get involved in tech? Is a computer science degree required?
  \item Is there a stage when more women seem to leave IT fields?
  \item Has the focus on recruiting and retaining women in computer science been effective?
  \item Why is it so important to get women involved in the first place?
\end{itemize}

This range of questions requires at least three types of data views:

\begin{enumerate}
  \item An overview, showing women entering or leaving computer science at each stage of the traditional IT pipeline.
  \item Detailed views, providing more details or supplementary information about each stage.
  \item A time series view, indicating the trends over time for at least one data series.
\end{enumerate}

Users should be able to navigate easily between the three types of views, as well as between the detailed views.
